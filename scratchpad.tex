\section{Appendix}\label{sec:appendix}

\subsection{Proofs}\label{sec:proofs}

\subsection{Combinatorial Laplacians}\label{sec:laplacian_theory}

The natural extension of the graph Laplacian \(L\) to simplicial complexes is the \emph{\(p\)-th combinatorial Laplacian} \(\Delta_{p}\), whose explicit matrix representation is given by {[}eq{]}:comb\_lap. Indeed, when \(p = 0\), \(\Delta_{0}(K) = \partial_{1}\partial_{1}^{T} = L\) recovers the graph Laplacian. As with boundary operators, \(\Delta_{p}(K)\) encodes simplicial homology groups in its nullspace, a result known as the discrete Hodge Theorem {[}lim2020hodge{]}:

\[{\widetilde{H}}_{p}\left( K;\mathbb{R} \right) \simeq \ker\left( \Delta_{p}(K) \right),\quad\beta_{p} = \operatorname{nullity}\left( \Delta_{p}(K) \right)\]

The fact that the Betti numbers of \(K\) may be recovered via the nullity of \(\Delta_{p}(K)\) has been well studied (see e.g. Proposition 2.2 of {[}horak2013spectra{]}). In fact, as pointed out by {[}horak2013spectra{]}, one need not only consider \(\Delta_{p}\) as the spectra of \(\Delta_{p}\), \(L_{p}^{\operatorname{up}}\), and \(L_{p}^{\operatorname{dn}}\) are intrinsically related by the identities:

\[\Lambda\left( \Delta_{p}(K) \right) \doteq \Lambda\left. \left( L_{p}^{\operatorname{up}} \right) \right. ⊍ \Lambda\left. \left( L_{p}^{\operatorname{dn}} \right) \right.,\quad\quad\Lambda\left. \left( L_{p}^{\operatorname{up}} \right) \right. \doteq \Lambda\left. \left( L_{p + 1}^{\operatorname{dn}} \right) \right.\]

where \(A \doteq B\) and \(A ⊍ B\) denotes equivalence and union between the \emph{non-zero} elements of the multisets \(A\) and \(B\), respectively. Moreover, all three operators \(\Delta_{p}\), \(L_{p}^{\operatorname{up}}\), and \(L_{p}^{\operatorname{dn}}\) are symmetric, positive semidefinite, and compact---thus, for the purpose of estimating \(\beta_{p}\), it suffices to consider only one family of operators.

\subsection{Laplacian matvec}\label{app:lap_matvec}

Given a simple undirected graph \(G = (V,E)\), let \(A \in \{ 0,1\}^{n \times n}\) denote its binary adjacency matrix satisfying \(A\left. \lbrack i,j\rbrack \right. = 1 \Leftrightarrow i \sim j\) if the vertices \(v_{i},v_{j} \in V\) are adjacent in \(G\), and let \(D = \operatorname{diag}\left. \left( \{\,\deg\left( v_{i} \right)\,\} \right) \right.\) denote the diagonal \emph{degree} matrix, where \(\deg\left( v_{i} \right) = \sum_{j \neq i}A\left. \lbrack i,j\rbrack \right.\). The \emph{graph Laplacian}`s adjacency, incidence, and element-wise definitions are:

\[L = D - A = \partial_{1} \circ \partial_{1}^{T}\,,\quad\quad L\,\left. \lbrack i,j\rbrack \right. = \left. \ \{\begin{array}{ll}
\deg\left( v_{i} \right) & \text{ if }i = j \\
 - 1 & \text{ if }i \sim j \\
0 & \text{ if }i \nsim j
\end{array} \right.\]

By using the adjacency relation \(i \sim j\) as in {[}chung1997spectral{]}, the linear and quadratic forms of \(L\) may be succinctly expressed as:

\[{L(x)}_{i} = \deg(v_{i}) \cdot x_{i} - \sum_{i \sim j}x_{j},\quad x^{T}Lx = \sum_{i \sim j}\left( x_{i} - x_{j} \right)^{2}\] \phantomsection\label{eq:lap_quad_form}{}

If \(G\) has \(m\) edges and \(n\) vertices taking labels in the set \(\left. \lbrack n\rbrack \right.\), observe computing the product from {[}eq{]}:lap\_quad\_form requires just \(O(m)\) time and \(O(n)\) storage via two edge traversals: one to accumulate vertex degrees and one to remove components from incident edges. By precomputing the degrees, the operation reduces further to a single \(O(n)\) product and \(O(m)\) edge pass, which is useful when repeated evaluations for varying values of \(x\) are necessary.

To extend the two-pass algorithm outlined above for \(p > 0\), we first require a generalization of the connected relation \(i \sim j\) from {[}eq{]}:lap\_quad\_form. Denote with \(\operatorname{co}(\tau) = \{\,\sigma \in K^{p + 1} \mid \tau \subset \sigma\,\}\) the set of proper cofaces of \(\tau \in K^{p}\), or \emph{cofacets}, and the (weighted) \emph{degree} of \(\tau \in K^{p}\) with: \[\deg_{w}(\tau) = \sum_{\sigma \in \operatorname{co}(\tau)}w(\sigma)\] Note setting \(w(\sigma) = 1\) for all \(\sigma \in K\) recovers the integral notion of degree representing the number of cofacets a given \(p\)-simplex has. Now, since \(K\) is a simplicial complex, if the faces \(\tau,\tau'\) share a common cofacet \(\sigma \in K^{p + 1}\), this cofacet \(\sigma\) is in fact \emph{unique} in the sense that \(\{\sigma\} = \operatorname{co}(\tau) \cap \operatorname{co}(\tau')\) (see {[}goldberg2002combinatorial{]} for more details). Thus, we may use a relation \(\tau\overset{\sigma}{\sim}\tau'\) to rewrite \(L_{p}^{\operatorname{up}}\) element-wise:

\[L_{p}^{\operatorname{up}}(\tau,\tau') = \left. \ \{\begin{array}{ll}
\deg_{w}(\tau) \cdot w^{+}(\tau) & \text{ if }\tau = \tau' \\
s_{\tau,\tau'} \cdot w^{+ /2}(\tau) \cdot w(\sigma) \cdot w^{+ /2}(\tau') & \text{ if }\tau\overset{\sigma}{\sim}\tau' \\
0 & \text{ otherwise}
\end{array} \right.\]

where \(s_{\tau,\tau'} = \operatorname{sgn}\left. \left( \left. \lbrack\tau\rbrack \right.,\partial\left. \lbrack\sigma\rbrack \right. \right) \right.\, \cdot \,\operatorname{sgn}\left. \left( \left. \lbrack\tau\rbrack \right.,\partial\left. \lbrack\sigma\rbrack \right. \right) \right.\). Ordering the \(p\)-faces \(\tau \in K^{p}\) along a total order and choosing an indexing function \(h:K^{p} \rightarrow \left. \lbrack n\rbrack \right.\) enables explicit computation of the corresponding matrix-vector product:

\[\left. \left( L_{p}^{\operatorname{up}}\, x \right) \right._{i} = \deg_{w}\left( \tau_{i} \right) \cdot w^{+}\left( \tau_{i} \right) \cdot x_{i} + w^{+ /2}\left( \tau_{i} \right)\sum_{\tau_{j}\overset{\sigma}{\sim}\tau_{i}}s_{\tau_{i},\tau_{j}} \cdot x_{j} \cdot w(\sigma) \cdot w^{+ /2}\left( \tau_{j} \right)\] \phantomsection\label{eq:l_up_matvec}{}

Observe \(v \rightarrow L_{p}^{\operatorname{up}}v\) can be evaluated now via a very similar two-pass algorithm as described for the graph Laplacian by simply enumerating the boundary chains of the simplices of \(K^{p + 1}\).

Below is pseudocode showing how to evaluate a weighted (up) Laplacian matrix-vector multiplication built from a simplicial complex \(K\) with \(m = \left| K^{p + 1} \right|\) and \(n = \left| K^{p} \right|\) in \(O(m)\) time when \(m > n\), assuming \(p\) is considered a small constant. Key to the runtime of the linear runtime operation is the constant-time determination of orientation between \(p\)-faces (\(s_{\tau,\tau'}\)) and---for sparse complexes---the use of a deterministic \(O(1)\) hash table \(h:K^{p} \rightarrow \left. \lbrack n\rbrack \right.\) for efficiently determining the appropriate input/output offsets (\(i\) and \(j\)).

In general, the signs of the coefficients \(\operatorname{sgn}\left. \left( \left. \lbrack\tau\rbrack \right.,\partial\left. \lbrack\sigma\rbrack \right. \right) \right.\) and \(\operatorname{sgn}\left. \left( \left. \lbrack\tau'\rbrack \right.,\partial\left. \lbrack\sigma\rbrack \right. \right) \right.\) depend on the position of \(\tau,\tau'\) as summands in \(\partial\left. \lbrack\sigma\rbrack \right.\), which itself depends on the orientation of \(\left. \lbrack\sigma\rbrack \right.\). Thus, evaluation of these sign terms takes \(O(p)\) time to determine for a given \(\tau \in \partial\left. \lbrack\sigma\rbrack \right.\) with \(\dim(\sigma) = p\), which if done naively via line (12) in the pseudocode {[}alg{]}:lap\_matvec increases the complexity of the algorithm. However, observe that the sign of their product is in fact invariant in the orientation of \(\left. \lbrack\sigma\rbrack \right.\) (see Remark 3.2.1 of {[}goldberg2002combinatorial{]})---thus, if we fix the orientation of the simplices of \(K^{p}\), the sign pattern \(s_{\tau,\tau'}\) for every \(\tau\overset{\sigma}{\sim}\tau'\) can be precomputed and stored ahead of time, reducing the evaluation \(s_{\tau,\tau'}\) to \(O(1)\) time and \(O(m)\) storage. Alternatively, if the labels of the \(p + 1\) simplices \(\sigma \in K^{p + 1}\) are given an orientation induced from the total order on \(V\) and \(p\) is a small constant, we can remove the storage requirement entirely and simply fix the sign pattern during the computation.

A subtle but important aspect of algorithmically evaluating {[}eq{]}:l\_up\_matvec is the choice of indexing function \(h:K^{p} \rightarrow \left. \lbrack n\rbrack \right.\). This map is necessary to deduce the contributions of the components \(x_{\ast}\) during the operation (line (13)). While this task may seem trivial as one may use any standard associative array to generate this map, typical implementations that rely on collision-resolution schemes such as open addressing or chaining only have \(O(1)\) lookup time in expectation. Moreover, empirical testing suggests that line (13) in {[}alg{]}:lap\_matvec can easily bottleneck the entire computation due to the scattered memory access such collision-resolution schemes may involve. One solution avoiding these collision resolution schemes that exploits the fact that \(K\) is fixed is to build an order-preserving \emph{perfect minimal hash function} (PMHF) \(h:K^{p} \rightarrow \left. \lbrack n\rbrack \right.\). It is known how to build PMHFs over fixed input sets of size \(n\) in \(O(n)\) time and \(O\left( n\log m \right)\) bits with deterministic \(O(1)\) access time {[}botelho2005practical{]}. Note that this process happens only once for a fixed simplicial complex \(K\): once \(h\) has been constructed, it is fixed for every \(\mathtt{matvec}\) operation.

\subsection{\texorpdfstring{Choosing a weight function }{Choosing a weight function }}

Given {[}eq{]}:inner\_product\_cochain, a natural question to ask whether there exists a weight function \(w\) that is more appropriate for rank computations. Since different weight choices yield different eigenvalue distribution in \(\mathcal{L}_{p}\), the ideal choice is one that is the most amenable to compute. For the purpose of improving the condition number of the underlying Laplacian, consider the following optimization problem:

\[\begin{aligned}
\max\limits_{w \in \mathbb{R}^{n}} & \lambda_{\min}\left( \mathcal{L}_{p}(w) \right) \\
\text{subject to} & w > 0,\:\mathbf{1}^{T}w = 1
\end{aligned}\] \phantomsection\label{eq:alg_connectivity}{}

In the spectral graph theory, a specialization of this problem (for \(p = 1\)) arises under the guise of related problems, such as maximizing \emph{algebraic connectivity} under a fixed total edge weight or finding the ``fastest mixing time'' of a Markov process. Fortunately, {[}eq{]}:alg\_connectivity is a convex optimization problem, which can be formulated as a semi-definite program (SDP) with variables \(\gamma,\beta \in \mathbb{R},w \in \mathbb{R}^{n}\) under positivity and sum-to-one constraints:

\[\begin{aligned}
\max\limits_{\gamma \in \mathbb{R}}\: & \gamma \\
\text{subject to}\: & \gamma I \preceq \partial_{1}D_{1}(w)\partial_{1}^{T} + \beta\mathbf{1}\mathbf{1}^{T},\quad w > 0,\quad\mathbf{1}^{T}w = 1
\end{aligned}\]

In general, we do not know if this approach generalizes for higher-order Laplacians, but in practice we found the weight function \(w^{\ast}\) that optimizes {[}eq{]}:alg\_connectivity to indeed be the easiest to approximate.
